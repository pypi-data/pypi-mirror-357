% --- START OF FILE deployment_en.tex ---

\documentclass[11pt, a4paper]{article}

% --- GENERAL PACKAGES AND CONFIGURATION ---
\usepackage[utf8]{inputenc}
\usepackage[T1]{fontenc}
\usepackage[english]{babel} % Changed to English
\usepackage{geometry}
\geometry{a4paper, margin=1in}

% --- FONTS AND TITLES ---
\usepackage{lmodern} % Modern font
\usepackage{titlesec}
\titleformat{\section}{\normalfont\Large\bfseries}{\thesection}{1em}{}
\titleformat{\subsection}{\normalfont\large\bfseries}{\thesubsection}{1em}{}
\titlespacing*{\section}{0pt}{3.5ex plus 1ex minus .2ex}{2.3ex plus .2ex}

% --- CODE FORMATTING (LISTINGS PACKAGE) ---
\usepackage{xcolor}
\usepackage{listings}

\definecolor{codegray}{rgb}{0.5,0.5,0.5}
\definecolor{codepurple}{rgb}{0.58,0,0.82}
\definecolor{codebackground}{rgb}{0.96,0.96,0.96}

% Style for shell commands (bash)
\lstdefinestyle{bashstyle}{
    backgroundcolor=\color{codebackground},   
    commentstyle=\color{codegray},
    keywordstyle=\color{blue},
    stringstyle=\color{codepurple},
    basicstyle=\ttfamily\small,
    breakatwhitespace=false,         
    breaklines=true,                 
    captionpos=b,                    
    keepspaces=true,                 
    numbers=left,                    
    numbersep=5pt,                   
    showspaces=false,                
    showstringspaces=false,
    showtabs=false,                  
    tabsize=2,
    frame=single,
    rulecolor=\color{black!20},
    language=bash,
    upquote=true,
}

% Style for Python
\lstdefinestyle{pythonstyle}{
    style=bashstyle,
    language=Python
}

% --- OTHER PACKAGES ---
\usepackage{hyperref}
\hypersetup{
    colorlinks=true,
    linkcolor=blue,
    filecolor=magenta,      
    urlcolor=cyan,
    pdftitle={GenPPI Deployment Guide for PyPI}, % Translated
    pdfpagemode=FullScreen,
}

% --- DOCUMENT START ---

\title{\bfseries GenPPI Deployment Guide for PyPI}
\author{GenPPI Team}
\date{\today}

\begin{document}
\maketitle

This document is a detailed guide on how to prepare, test, and publish the GenPPI Python interface on the Python Package Index (PyPI). Version 0.1.5 uses the py7zr library for multi-volume file extraction to ensure compatibility.

\section{Preparation Checklist}

Before publishing, follow this checklist to ensure the package is ready.

\subsection{Check Package Structure}
Your project structure should look like this:
\begin{verbatim}
genppi_py/
|-- dev.py
|-- docs
|   |-- deployment.pdf
|   |-- deployment.tex
|   |-- README.md
|   |-- testing.pdf
|   `-- testing.tex
|-- genppi_py
|   |-- bin
|   |   `-- __init__.py
|   |-- cli.py
|   |-- download_model_direct.py
|   |-- download_model.py
|   |-- download_samples.py
|   |-- genppi.py
|   `-- __init__.py
|-- install.bat
|-- install.sh
|-- MAINTAINERS.md
|-- MANIFEST.in
|-- pyproject.toml
|-- README.md
|-- scripts
|   |-- build_and_test.py
|   |-- deploy_production.py
|   |-- deploy_test.py
|   |-- quick_check.py
|   |-- README.md
|   |-- test_installation.py
|   `-- test_upload.py
|-- setup.py
|-- tests
|   |-- README.md
|   `-- test_genppi.py
`-- tools
    |-- prepare_release.py
    |-- README.md
    |-- requirements-dev.txt
    `-- tox.ini
\end{verbatim}

\subsection{Update Package Version}
Edit the \texttt{genppi\_py/\_\_init\_\_.py} and \texttt{pyproject.toml} files to update the version number, following semantic versioning.
\begin{lstlisting}[style=pythonstyle]
# in genppi_py/__init__.py
__version__ = '0.1.5'  # Current version

# in pyproject.toml
[project]
name = "genppi-py"
version = "0.1.5"
# ... rest of the configuration
\end{lstlisting}

\subsection{Review Configuration Files}
\subsubsection{setup.py}
Version 0.1.5 uses a simplified configuration. Required dependencies are defined in \texttt{pyproject.toml}:
\begin{lstlisting}[style=pythonstyle]
# Current setup.py configuration (simplified)
setup(
    cmdclass={
        'install': CustomInstall,
    },
)

# Dependencies are defined in pyproject.toml
[project]
dependencies = [
    "py7zr>=0.20.0,<1.0.0",  # For multi-volume 7z extraction
    "multivolumefile>=0.2.3",  # Required by py7zr
]
\end{lstlisting}
Version 0.1.5 ensures that critical dependencies (py7zr and multivolumefile) are installed automatically. This eliminates issues with multi-volume file extraction.

\textbf{Important notes on licenses:} 
\begin{itemize}
    \item Ensure the package name in \texttt{README.md} matches the name in \texttt{setup.py} (\texttt{genppi-py}).
    \item \textbf{Issue with modern license fields:} TestPyPI does not support fields like \texttt{license\_files} or \texttt{license-expression}. To avoid upload errors, use only the license classifier in \texttt{classifiers}.
    \item If you encounter a \texttt{"unrecognized or malformed field 'license-file'"} error, temporarily remove the \texttt{LICENSE} file during the build, or just use the classifier.
    \item For maximum compatibility, only use: \texttt{'License :: OSI Approved :: GNU General Public License v3 (GPLv3)'} in the classifiers.
\end{itemize}

\subsubsection{pyproject.toml}
This file contains the modern project configuration, including metadata and dependencies:
\begin{lstlisting}[style=bashstyle]
[build-system]
requires = ["setuptools>=42", "wheel"]
build-backend = "setuptools.build_meta"

[project]
name = "genppi-py"
version = "0.1.5"
description = "Python interface for GenPPI"
dependencies = [
    "py7zr>=0.20.0,<1.0.0",
    "multivolumefile>=0.2.3",
]

[project.scripts]
"genppi" = "genppi_py.genppi:main"
"genppi-download-samples" = "genppi_py.download_samples:download_samples"
"genppi-download-model" = "genppi_py.download_model:main"
\end{lstlisting}

\subsection{Generate Distribution Files}
Install the build tools and create the distribution packages.
\begin{lstlisting}[style=bashstyle]
# Install the tools if you haven't already
pip install build twine

# Clean old builds to avoid issues
rm -rf dist/ build/ *.egg-info

# Create the new distribution packages
python -m build
\end{lstlisting}
This command will create a \texttt{dist/} folder containing a \texttt{.tar.gz} file (source distribution) and a \texttt{.whl} file (wheel).

\section{Publishing to PyPI}
With the packages generated, the next step is to publish them.

\subsection{Step 1: Publish to TestPyPI (Highly Recommended)}
Always publish to the test server first to ensure everything works.
\begin{lstlisting}[style=bashstyle]
# Upload to TestPyPI
python -m twine upload --repository testpypi dist/*
\end{lstlisting}
You will need an account on \href{https://test.pypi.org/}{TestPyPI}. After uploading, test the installation in a clean virtual environment:
\begin{lstlisting}[style=bashstyle]
# Create and activate a new virtual environment
python -m venv test_env
source test_env/bin/activate

# Install version 0.1.5 from TestPyPI (with dependencies from official PyPI)
pip install --index-url https://test.pypi.org/simple/ --extra-index-url https://pypi.org/simple/ genppi-py==0.1.5

# Verify critical dependencies
python -c "import py7zr; print('py7zr:', py7zr.__version__)"
python -c "import multivolumefile; print('multivolumefile OK')"

# Run a quick test
genppi --help
\end{lstlisting}

\subsection{Step 2: Publish to the Official PyPI}

\textbf{IMPORTANT - TestPyPI vs. Official PyPI:}

\textbf{TestPyPI (For Developers Only):}
\begin{itemize}
    \item A test environment with limited dependencies.
    \item Requires a complex command to install dependencies.
    \item End-users should NOT use this command.
\end{itemize}

\textbf{Official PyPI (For End-Users):}
\begin{itemize}
    \item A production environment with all dependencies.
    \item Uses a simple command for installation.
    \item Provides an optimized user experience.
\end{itemize}

If the TestPyPI trial was successful, publish to the official PyPI:
\begin{lstlisting}[style=bashstyle]
# Upload to the official PyPI
python -m twine upload dist/*

# Or use an automated script
python deploy_production.py
\end{lstlisting}

\textbf{Result for the End-User:}
After the official release, users can install simply with:
\begin{lstlisting}[style=bashstyle]
# SIMPLE COMMAND FOR END-USERS
pip install genppi-py
\end{lstlisting}

You will need an account on \href{https://pypi.org/}{PyPI}.

\subsection{Step 3: Final Verification}
After publishing, check your project page at \texttt{https://pypi.org/project/genppi-py/}. Finally, perform one last installation test from the official PyPI in a clean virtual environment to ensure a smooth end-user experience.

\subsection{Common Troubleshooting}
\subsubsection{Error: \texttt{unrecognized or malformed field 'license-file'}}
This error occurs when setuptools generates modern license fields that PyPI does not yet support. To resolve it:
\begin{enumerate}
    \item Temporarily remove the \texttt{LICENSE} file from the project directory.
    \item Use only the traditional license classifier in \texttt{setup.py}.
    \item Avoid using \texttt{license\_files} or license fields in \texttt{pyproject.toml}.
    \item After a successful upload, restore the \texttt{LICENSE} file.
\end{enumerate}

\subsubsection{Verifying Metadata}
Before uploading, always check the generated metadata:
\begin{lstlisting}[style=bashstyle]
# Check the metadata content
cat genppi_py.egg-info/PKG-INFO | head -20

# Look for problematic fields such as:
# License-File: LICENSE
# License-Expression: GPL-3.0-or-later
\end{lstlisting}

\subsubsection{Dependency Issues on TestPyPI}
Version 0.1.5 uses dependencies available on the official PyPI. If you encounter issues:
\begin{lstlisting}[style=bashstyle]
# Check if py7zr is available
pip install "py7zr>=0.20.0,<1.0.0"
pip install "multivolumefile>=0.2.3"

# Then install the test package (with dependencies from official PyPI)
pip install --index-url https://test.pypi.org/simple/ --extra-index-url https://pypi.org/simple/ genppi-py==0.1.5

# Verify installation
python -c "import py7zr, multivolumefile; print('Dependencies OK')"
\end{lstlisting}

\section{Automation and Maintenance}

\subsection{Automated Deployment Scripts}
The project includes automated scripts to streamline the deployment process:

\subsubsection{quick\_check.py}
Performs a quick check of dependencies and features:
\begin{lstlisting}[style=bashstyle]
python quick_check.py
\end{lstlisting}

\subsubsection{build\_and\_test.py}
Runs a full build with comprehensive tests:
\begin{lstlisting}[style=bashstyle]
python build_and_test.py
\end{lstlisting}

\subsubsection{deploy\_test.py}
Automates deployment to the test environment:
\begin{lstlisting}[style=bashstyle]
python deploy_test.py
\end{lstlisting}

\subsection{Manual Publishing Script}
For manual releases, you can use a script like \texttt{publish.sh}:
\begin{lstlisting}[style=bashstyle]
#!/bin/bash
set -e # Exit script if a command fails

# Ask for version confirmation
read -p "Have you updated the version in __init__.py? (y/n) " -n 1 -r
echo
if [[ ! $REPLY =~ ^[Yy]$ ]]; then
    exit 1
fi

echo "Cleaning old builds..."
rm -rf dist/ build/ *.egg-info

echo "Generating new packages..."
python -m build

echo "Uploading to PyPI..."
python -m twine upload dist/*

echo "Publication complete!"
\end{lstlisting}

\subsection{Updating the Package}
To release a new version:
\begin{enumerate}
    \item Update the source code with new features or fixes.
    \item If needed, update the Lisp executables or \texttt{model.dat} file in the source GitHub repository.
    \item Increment the version number in \texttt{genppi\_py/\_\_init\_\_.py}.
    \item Run the \texttt{publish.sh} script to publish the new version.
\end{enumerate}

\subsection{Updating \texttt{model.dat}}
If you need to generate new versions of \texttt{model.dat}:
\begin{enumerate}
    \item Create the new \texttt{model.dat} file.
    \item Compress it into parts using 7-Zip (e.g., 10MB parts):
    \begin{lstlisting}[style=bashstyle]
7z a -v10m model.7z model.dat
    \end{lstlisting}
    \item Test the extraction with py7zr:
    \begin{lstlisting}[style=bashstyle]
python -c "
import py7zr
from multivolumefile import MultiVolume
with MultiVolume('model.7z', mode='rb') as vol:
    with py7zr.SevenZipFile(vol, mode='r') as archive:
        print('Files in volume:', archive.getnames())
        archive.extractall(path='test_extract')
"
    \end{lstlisting}
    \item Replace the \texttt{model.7z.00*} files in your source repository.
\end{enumerate}

\end{document}
% --- END OF FILE deployment_en.tex ---

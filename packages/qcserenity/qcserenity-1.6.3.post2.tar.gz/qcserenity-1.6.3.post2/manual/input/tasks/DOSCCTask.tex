\subsection{Task: DOS-Coupled Cluster}
This task combines the localization task, the generalized direct orbital selection (GDOS)
task, and the wavefunction embedding task. It allows the user to calculate relative energies
with a multi-level DLPNO-CC approach using the orbital sets generated by the GDOS.
\subsubsection{Example Input}
\begin{lstlisting}
+task DOSCC
  act reactant
  act product
  dosSettings normal
  +lc0
    method DLPNO-CCSD(T0)
    pnoSettings normal
    useFrozenCore true
  -lc0
  +lc1
    method DLPNO-CCSD
    pnoSettings loose
    useFrozenCore true
  -lc1
-task
\end{lstlisting}
\subsubsection{Basic Keywords}
\begin{description}
\item [\texttt{name}]\hfill \\
Aliases for this task are \ttt{DOSCC} and \ttt{DOSCCTask}.
\item [\texttt{activeSystems}]\hfill \\
Accepts a set of active systems as supersystems. Note that at least two systems need to be supplied.
All systems have to fulfill the criteria for the direct orbital selection (see task~\ref{sec:activeSpaceTask}).
\item [\texttt{sub-blocks}]\hfill \\
Local correlation (\ttt{lc}), localization task settings (\ttt{loc}). and non-blocked settings
of the wavefunction embedding task (\ttt{wfemb}) are added via sub-blocks in the task settings.
The local correlation settings are added as a list of settings. For each fragment (environment) system there is
a local correlation settings object that determines the settings employed for the orbitals of this system. The
task assumes that systems/settings with a lower index are tighter.
By default \ttt{splitValenceAndCore = true} is set for the localization task settings.
\item [\texttt{alignOrbitals}]\hfill \\
If true, the orbitals of each system are aligned with respect to the first system orbitals. By default \ttt{true}.
\item [\texttt{dosSettings}]\hfill \\
Macro flag to select multiple DOS thresholds (see task~\ref{task:gdos}, \ttt{similarityLocThreshold}
and \ttt{similarityKinEnergyThreshold}). The flag \ttt{LOOSE} corresponds to the threshold triple \ttt{\{1e-1, 1e-2, 1e-3\}},
\ttt{NORMAL} to \ttt{\{5e-2, 5e-3, 1e-3\}}, \ttt{TIGHT} to  \ttt{\{2e-2, 2e-3, 8e-4\}}, \ttt{VERYTIGHT} to \ttt{\{8e-3, 1e-3, 1e-4\}},
and \ttt{EXTREME} to \ttt{\{5e-3, 5e-4, 1e-4\}}.
By default \ttt{NORMAL}. The thresholds set by the flag may be overridden manually.
\item [\texttt{printGroupAnalysis}]\hfill \\
If true, print information on orbitals pairs. By default \ttt{true}.
\item[\texttt{pairCutoff}] \hfill \\
Vector of cutoffs to determine what pair belongs to which settings based on energy differences. The default is \ttt{\{1e-4, 0.0\}}. The vector is assumed to have the tighter settings at smaller indices -  corresponding to higher pair cutoffs.
\item[\texttt{normThreshold}] \hfill \\
Maximum entry left in the residual matrix / vector in order to assume that the amplitude optimization in a DLPNO-CCSD calculation is converged. \ttt{1e-5} by default. See section~\ref{sec:coupledClusterTask} for further information.
\item[\texttt{maxCycles}] \hfill \\
Maximum number of iterations allowed in the amplitude optimization in a DLPNO-CCSD calculation. \ttt{100} by default. See section~\ref{sec:coupledClusterTask} for further information.
\item[\texttt{strictPairEnergyThreshold}] \hfill \\
If true, choose the tightest pair threshold given for the CCSD pair truncation. \ttt{True} by default.
\item[\texttt{strictTriples}] \hfill \\
If true, the triples construction of the orbital pair will be determined based on the tightest settings of the group rather than the orbital pair's settings. \ttt{True} by default.
\item[\texttt{skipCrudePresPairSelected}] \hfill \\
If true, the crude SC-MP2 prescreening step in a DLPNO-CCSD calculation will be skipped. \ttt{False} by default.  
 
\end{description}

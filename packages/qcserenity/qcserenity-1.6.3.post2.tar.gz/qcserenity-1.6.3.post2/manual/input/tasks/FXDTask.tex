\subsection{Task: FXD}
\label{sec:FXD}
This task performs fragment charge difference (FCD), fragment excitation difference (FED) and multistate FED-FCD diabatization calculations.
So far, this task assumes that the previous calculation was a TDA/CIS-type calculation. 
\subsubsection{Example Input}
\begin{lstlisting}
+task FXD
  act system
  donoratoms {start end}
  acceptoratoms {start end}
  fed true
-task
\end{lstlisting}
\subsubsection{Basic Keywords}
\begin{description}
	\item [\texttt{activeSystem}]\hfill \\
	The system for which the diabatization procedure is performed.
	\item [\texttt{donorAtoms}]\hfill \\
	Specifying the atoms belonging to the donor fragment \ttt{\{start end\}}. Starts with zero. 
	\item [\texttt{acceptorAtoms}]\hfill \\
	Specifying the atoms belonging to the donor fragment \ttt{\{start end\}}. 
	\item [\texttt{FED}]\hfill \\
	If \texttt{true} performs a fragment excitation difference calculation. The default is \ttt{false}.
	\item [\texttt{FCD}]\hfill \\
	If \texttt{true} performs a fragment charge difference calculation. The default is \ttt{false}.
	\item [\texttt{multistateFXD}]\hfill \\
	If \texttt{true} performs a multistate FED-FCD. The default is \ttt{false}.
	\item [\texttt{states}]\hfill \\
  Specifies the number for excited states used for the diabatization procedure. The default is \ttt{100}.
	\item [\texttt{loewdinPopulation}]\hfill \\
	Whether a Löwdin populations are used for the diabatization. The default is \ttt{true}. If it is set to \ttt{false}, Mulliken populations are used.
	\item [\texttt{writeTransformedExcitationVectors}]\hfill \\
	Writes the diabatic excitation vector to disk. The default is \ttt{false}. This is needed to plot the corresponding NTOs.
\end{description}

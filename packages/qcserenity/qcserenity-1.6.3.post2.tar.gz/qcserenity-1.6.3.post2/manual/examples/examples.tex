\chapter{Example Inputs}
\section{Standard SCF}
The first example is a minimal input for a restricted HF single point calculation.
\begin{lstlisting}
+system
  name watera
  geometry watera.xyz
  method hf
-system

+task scf
  act watera
-task
\end{lstlisting}
For systems with a spin not equal to zero the calculation will automatically be run in the \ttt{unrestricted} mode
as in the U-KS-DFT example below.
\begin{lstlisting}
+system
  name watera
  geometry watera.xyz
  method dft
  spin 1
  charge -1
  +dft
    functional pbe0
  -dft
  +basis
    label def2-TZVP
  -basis
-system

+task scf
  act watera
-task
\end{lstlisting}
For systems with a spin equal to zero the unrestricted mode can be forced, using the appropriate keyword,
as in the final example of this section:
\begin{lstlisting}
+system
  name watera
  geometry watera.xyz
  method dft
  scfmode unrestricted
  +dft
    functional pbe0
  -dft
  +basis
    label def2-TZVP
  -basis
-system

+task scf
  act watera
-task
\end{lstlisting}
A continuum solvation model can be enabled by changing the flag 'use' in the PCM-block to true:
\begin{lstlisting}
+system
  name watera
  geometry watera.xyz
  method dft
  scfmode restricted
  +dft
    functional pbe0
  -dft
  +basis
    label def2-TZVP
  -basis
  +pcm
    use true
    solverType iefpcm
    solvent water
    cavity delley
  -pcm
-system

+task scf
  act watera
-task
\end{lstlisting}
Density-fitting for different contributions can be controlled with the appropriate keywords in the Basis-block. If Cholesky related methods are used they can also be controlled in that block:
\begin{lstlisting}
  +system
    name watera
    geometry watera.xyz
    method dft
    scfmode restricted
    +dft
      functional cam-b3lyp
    -dft
    +basis
      label def2-TZVP
      densfitJ RI
      densfitK NONE
      densfitLRK ACD
      cdThreshold 1e-6
      secondCD 1e-8
      extendSphericalACDShells COMPLETE
    -basis
  -system
  
  +task scf
    act watera
  -task
  \end{lstlisting}

\section{Coupled Cluster Calculation}
This example shows how to perform a (frozen core) DLPNO-CCSD calculation:
\begin{lstlisting}
+system
  name watera
  geometry watera.xyz
  method hf
-system

+task scf
  act watera
-task

+task loc
  act watera
  locType IBO
  splitValenceAndCore true
-task

+task CC
  act watera
  level DLPNO-CCSD
  +LC
    useFrozenCore true
    pnoSettings TIGHT
  -LC
-task
\end{lstlisting}

\section{Geometry Optimization}
This example shows how to optimize a structure using DFT:
\begin{lstlisting}
+system
  name watera
  geometry watera.xyz
  method dft
  +dft
    functional PBE
    dispersion d3bjabc
  -dft
-system

+task opt
  act watera
  maxCycles 100
  tightOpt true
-task
\end{lstlisting}
in case of a sDFT optimization extra sDFT options (\ttt{naddXCFunc},\ttt{naddKinFunc}) and the names of the other system(s) (\ttt{waterb}) have to be added.
\begin{lstlisting}
+task opt
  act watera
  act waterb
  +EMB
    naddXCFunc PBE
    dispersion d3bjabc
    naddKinFunc LLP91K
  -EMB
  maxCycles 100
  tightOpt true
-task
\end{lstlisting}

\section{Hessian Calculation}
This example shows how to calculate a (sermi-numerical) Hessian using DFT:
\begin{lstlisting}
+system
  name watera
  geometry watera.xyz
  method dft
  +dft
    functional PBE
    dispersion d3bjabc
  -dft
-system

+task hess
  act watera
-task

\end{lstlisting}
in case of a sDFT run extra sDFT options (\ttt{naddXCFunc}, \ttt{naddKinFunc}) and the names of the other system(s) (\ttt{waterb}) have to be added.
\begin{lstlisting}
+task hess
  act watera
  act waterb
  +EMB
    dispersion d3bjabc
    naddXCFunc PBE
    naddKinFunc LLP91K
  -EMB
-task
\end{lstlisting}


\section{Frozen Density Embedding (FDE)}
A minimal input of a DFT calculation embedded in the frozen density of another system.
The environment system will be calculated in an isolated manner implicitly.
The basis is kept to be the default. Note that it is in principle also possible to have
both system run using Hartree--Fock and only the interaction \textit{via} FDE.
\begin{lstlisting}
+system
  name watera
  geometry watera.xyz
  method dft
  +dft
    functional pw91
  -dft
-system

+system
  name waterb
  geometry waterb.xyz
  method dft
  +dft
    functional pw91
  -dft
-system

+task fde
  act watera
  env waterb
  +EMB
    naddxcfunc pw91
    naddkinfunc pw91k
  -EMB
-task
\end{lstlisting}
A continuum solvation model enclosing all subsystems can be used by setting 'use' to true
in the task specific PCM-block. Note that any PCM-settings set for any subsystem are ignored
in the embedding step of all FDE-type calculations. However, they will affect the isolated
SCF calculations for the subsystems.
\begin{lstlisting}
+system
  name watera
  geometry watera.xyz
  method dft
  +dft
    functional pw91
  -dft
-system

+system
  name waterb
  geometry waterb.xyz
  method dft
  +dft
    functional pw91
  -dft
-system

+task fde
  act watera
  env waterb
  +EMB
    naddxcfunc pw91
    naddkinfunc pw91k
  -EMB
  +pcm
    use true
    solverType cpcm
    solvent water
    cavity delley
  -pcm
-task
\end{lstlisting}

\section{Freeze-and-Thaw (FaT/sDFT)}
A minimal input for a \textit{freeze-and-thaw} (FaT/sDFT) calculation.
In addition to the two given systems which will be iterated over until
convergence, additional \ttt{environment} systems can be added, which will never
be relaxed. More \ttt{active} systems are also possible, of course.
\begin{lstlisting}
+system
  name watera
  geometry watera.xyz
  method dft
  +dft
    functional pw91
  -dft
-system

+system
  name waterb
  geometry waterb.xyz
  method dft
  +dft
    functional pw91
  -dft
-system

+task fat
  act watera
  act waterb
  convThresh 1e-6
  +EMB
    naddxcfunc pw91
    naddkinfunc pw91k
  -EMB
-task
\end{lstlisting}
A continuum solvation model enclosing all subsystems can be used in the same way as for
FDE.

\section{MP2-in-DFT Embedding via Projection}
The following example will combine the two given fragments/subsystems into one bigger supersystem, and optimize its orbitals using
the settings of the environment system (KS-DFT). Afterward the orbitals will be localized and the active system will be picked,
then the active system will be optimized (using HF), while embedded within the environment orbitals \textit{via} projection based embedding.
Finally, the altered active system energy will be corrected using RI-MP2. Note that this example does not include a basis set truncation within
the projection-based embedding part.
\begin{lstlisting}
+system
  name watera
  geometry watera.xyz
  method HF
-system

+system
  name waterb
  geometry waterb.xyz
  method dft
  +dft
    functional PBE
  -dft
-system

+task PBE
  act watera
  env waterb
  +EMB
    naddxcfunc PBE
  -EMB
-task

+task MP2
 act watera
-task

\end{lstlisting}
A continuum solvation model enclosing both subsystems can be used in the same way as for
FDE.

\section{Direct Orbital Selection-based DLPNO-CCSD(T$_0$)-in-DFT Embedding via the Huzinaga Equation}
This example performs Direct Orbital Selection (DOS)-based DLPNO-CCSD(T$_0$)-in-DFT calculations for two points
(reactant and transition state) along a reaction coordinate. First the active orbital space for the two systems
is selected using the ActiveSpaceSelectionTask. Then the embedded calculations are performed in which the subsystem
orbitals are constrained to stay orthogonal using the shifted Huzinaga equation. Finally, the local coupled cluster
calculations are done for the active systems embedded into their respective environments.
\begin{lstlisting}
+system
  name Reactant
  geometry r.xyz
  method dft
-system

+system
  name Reactant_active
  geometry r.xyz
  method HF
-system

+system
  name Reactant_environment
  geometry r.xyz
  method dft
-system

+system
  name TransitionState
  geometry ts.xyz
  method dft
-system

+system
  name TransitionState_active
  geometry ts.xyz
  method HF
-system

+system
  name TransitionState_environment
  geometry ts.xyz
  method dft
-system

+task ACTIVESPACETASK
  super Reactant
  super TransitionState
  act Reactant_active
  act TransitionState_active
  env Reactant_environment
  env TransitionState_environment
  usePiBias true
  alignPiOrbitals true
-task

+task FDE
  act Reactant_active
  env Reactant_environment
  +EMB
    embeddingMode FERMI
  -EMB
-task

+task FDE
  act TransitionState_active
  env TransitionState_environment
  +EMB
    embeddingMode FERMI
  -EMB
-task

+task loc
  act Reactant_active
  locType IBO
  splitValenceAndCore true
-task

+task CC
  act Reactant_active
  env Reactant_environment
  level DLPNO-CCSD(T0)
  +EMB
    embeddingMode FERMI
  -EMB
  +LC
    pnoSettings TIGHT
  -LC
-task

+task loc
  act TransitionState_active
  locType IBO
  splitValenceAndCore true
-task

+task CC
  act TransitionState_active
  env TransitionState_environment
  level DLPNO-CCSD(T0)
  +EMB
    embeddingMode FERMI
  -EMB
  +LC
    pnoSettings TIGHT
  -LC
-task
\end{lstlisting}

\section{Potential Reconstruction (OEP)}
The following example input will run a freeze-and-thaw calculation employing accurate non-additive kinetic potentials generated using
potential reconstruction techniques (\ttt{embeddingmode reconstruction}). It will choose what we call to Bottom-Up approach (\cite{good2010}) (as it is the Freeze and Thaw Task).
However, the Top-Down approach (\cite{fux2010}) is also available (by choosing the TDEmbedding Task). A supersystem basis is used throughout
the freeze-and-thaw calculation (extendBasis true, basisExtThresh -1) in order to generate accurate total densities $\rho^\mathrm{tot}(\vec{r})$. During
the Wu--Yang potential reconstruction (\cite{wu2003}), the potential will be expressed in the def2-QZVP (\cite{weig2005}) basis. A pseudo inversion of the Hessian
will be performed, neglecting eigenvalues lower than $|\text{max. eigenvalue}|\cdot 10^{-5}$ ( \ttt{singValThreshold 1e-5}). Furthermore, in order to generate
physically meaningful potentials, a constraint as proposed in Ref. (\cite{heat2007}) is employed ( \ttt{smoothFactor 1e-3}).
\begin{lstlisting}
+system
 name ammonia1
 geometry ammonia1.xyz
 method dft
 +dft
  functional PW91
 -dft
-system

+system
 name ammonia2
 geometry ammonia2.xyz
 method dft
 +dft
  functional PW91
 -dft
-system

+task fat
  system ammonia1
  system ammonia2
  +EMB
    naddXcFunc PW91
    embeddingmode reconstruction
    potentialBasis def2-QZVP
    singValThreshold 1e-5
    smoothFactor 1e-3
  -EMB
  extendBasis true
  basisExtThresh -1
-task
\end{lstlisting}

\newpage
\section{LRSCF}
CIS (TDHF with TDA):
\begin{lstlisting}
+system
 name water
 geometry water.xyz
 method hf
 +basis
  label def2-SVP
 -basis
-system

+task scf
 act water
-task

+task lrscf
 act water
 method tda
 nEigen 5
-task
\end{lstlisting}
\newpage

ADC(2) (or CC2) with transition moments:
\begin{lstlisting}
+system
 name water
 geometry water.xyz
 method hf
 +basis
  label def2-SVP
 -basis
-system

+task scf
 act water
-task

+task lrscf
 act water
 method adc2 (or cc2)
 nEigen 5
 ccprops true
-task
\end{lstlisting}
\newpage

TDDFT Response Properties in the dipole-velocity representation:
\begin{lstlisting}
+system
 name water
 geometry water.xyz
 method dft
 +dft
  functional pbe0
 -dft
 +basis
  label def2-SVP
 -basis
-system

+task scf
 act water
-task

+task lrscf
 act water
 frequencies { 2 3 4 }
 gauge velocity
-task
\end{lstlisting}
\newpage

FDEu-TDDFT for \ttt{water1} in the environment of \ttt{water2}:
\begin{lstlisting}
+system
 name water1
 geometry water_monA.xyz
 method dft
 +dft
  functional PW91
 -dft
 +basis
  label def2-SVP
 -basis
-system

+system
 name water2
 geometry water_monB.xyz
 method dft
 +dft
  functional PW91
 -dft
 +basis
  label def2-SVP
 -basis
-system

+task fat
 act water1
 act water2
 +emb
 embeddingmode naddfunc
 naddxcfunc pw91
 naddkinfunc pw91k
 -emb
-task

+task lrscf
 act water1
 env water2
 nEigen 5
 +emb
 embeddingmode naddfunc
 naddxcfunc pw91
 naddkinfunc pw91k
 -emb
-task
\end{lstlisting}
\newpage

FDEc-TDDFT (coupling FDEu excitations of \ttt{water1} and \ttt{water2}). Specifically coupling
two excitations of the first subsystem (the first and the third) and three excitations on the second subsystem (the second, the third, and the fourth):
\begin{lstlisting}
+system
 name water1
 geometry water_monA.xyz
 method dft
 +dft
  functional PW91
 -dft
 +basis
  label def2-SVP
 -basis
-system

+system
 name water2
 geometry water_monB.xyz
 method dft
 +dft
  functional PW91
 -dft
 +basis
  label def2-SVP
 -basis
-system

+task fat
 act water1
 act water2
 +emb
 embeddingmode naddfunc
 naddxcfunc pw91
 naddkinfunc pw91k
 -emb
-task

+task lrscf
 act water1
 env water2
 nEigen 5
 +emb
 embeddingmode naddfunc
 naddxcfunc pw91
 naddkinfunc pw91k
 -emb
-task

+task lrscf
 act water2
 env water1
 nEigen 5
 +emb
 embeddingmode naddfunc
 naddxcfunc pw91
 naddkinfunc pw91k
 -emb
-task

+task lrscf
 act water1
 act water2
 uncoupledSubspace {2 1 3 3 2 3 4}
 +emb
 embeddingmode naddfunc
 naddxcfunc pw91
 naddkinfunc pw91k
 -emb
-task
\end{lstlisting}
\newpage
TDDFT gradients of the second and third excited states of a water molecule using a range-separated hybrid functional and the RI-J approximation (but only for the LRSCF part) and a finer grid.
\begin{lstlisting}
+system
 name water
 geometry water.xyz
 method dft
 +dft
  functional camb3lyp
 -dft
 +basis
  label def2-SVP
  densfitj none   # this applies to the SCF
  auxclabel ri-j-weigend
 -basis
-system

+task scf
 act water
-task

+task lrscf
 act water
 neigen 4
 excgradlist {2 3}
 densfitJ RI    # this applies to the LRSCF, uses the auxiliary basis defined as auxclabel
 +grid
  accuracy 7
  smallgridaccuracy 7
 -grid
-task
\end{lstlisting}
\newpage

\section{Restarting and Properties}
Below you can find a small input that reads the results of a previous sDFT run, and evaluates the total energy once more.
After that it calculates multipole moments for the subsystems, and Mulliken populations of one of the systems.
Note that the input will not run a single SCF cycle if all the data from the sDFT run is present.
\begin{lstlisting}
+system
 name watera
 load ./oldrun/watera
-system

+system
 name waterb
 load ./oldrun/waterb
-system

+task FaT
  act watera
  act waterb
  +EMB
    naddxcfunc pw91
    naddkinfunc pw91k
  -EMB
  maxCycles 0
-task

+task multipole
  act watera
-task

+task multipole
  act waterb
-task

+task pop
  act watera
  mulliken true
-task
\end{lstlisting}

\clearpage
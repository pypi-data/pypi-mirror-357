\chapter{Install}
Please read the following instructions carefully, these instructions can also be found in the \texttt{README.md} distributed with
the source code.

\section{Prerequisites}
The code has been tested and compiled on Linux with GCC/G++
(Versions 7 and newer) and Clang (Versions 8 and newer).
Compilation with other compilers such as ICC should be
possible on Linux.\\
\\
Furthermore, the code has been compiled on macOS using Clang
but may experience problems depending on the CPU architecture.
Compilation with GCC on macOS should most likely also be possible.\\
\\
Compilation on and for Windows is not supported at the moment.
\\
The following programs/libraries must be available on your system:
\begin{itemize}
 \item CMake (Version $>=$ 3.15)
 \item Boost (for package managers: including boost-devel)
 \item OpenMP
 \item Eigen3
 \item HDF5 (Version $>=$ 1.10.1; including header files and cmake files)
 \item A recent GMP version, including C++ support (for libint2)
 \item The standard GNU tool chain (make, tar, autoconf, libtool)
\end{itemize}
The following libraries will be automatically downloaded and installed together
with \serenity (unless \texttt{SERENITY\_DOWNLOAD\_DEPENDENCIES=OFF} is set):
\begin{itemize}
 \item libint2 (Version 2.7.0-beta6, pre-configured and hosted at: \url{https://www.uni-muenster.de/Chemie.oc/THCLAB/libint})
 \item libecpint (v1.0.7 from \url{https://github.com/robashaw/libecpint.git})
 \item libxc (v6.1.0 from \url{https://gitlab.com/libxc/libxc})
 \item xcfun (the Serenity version is forked to: \url{https://github.com/qcserenity/xcfun})
 \item GTest (Google Test and Google Mock, v1.13.0 from \url{https://github.com/google/googletest.git})
\end{itemize}
The following libraries are optional and needed for additional features:
\begin{itemize}
 \item Intel MKL (for SMP parallel Eigen3 eigenvalue solvers)
 \item Doxygen (for the documentation)
 \item Python-devel (for the Python wrapper)
 \item pybind11 (for the Python wrapper)
 \item laplace-minimax (for Laplace-Transform MP2/ADC(2)/CC2, commit: '55414f3', \url{https://github.com/bhelmichparis/laplace-minimax.git})
\end{itemize}

\section{Install Using CMake and Make}

Extract or pull the source code, then create a build directory:
\begin{lstlisting}
cd serenity
mkdir build
cd build
\end{lstlisting}
Then run \texttt{cmake}:
\begin{lstlisting}
cmake ..
\end{lstlisting}
To compile Serenity for your specific CPU architecture, you can add:
\begin{lstlisting}
cmake -DSERENITY_MARCH=native ..
\end{lstlisting}
(If the build folder is not located inside the main directory of \textsc{Serenity}
please adapt the path accordingly.)

Finally run \texttt{make} to build the program:
\begin{lstlisting}
make
\end{lstlisting}
Instead of additionally running \texttt{make install}, you can source \texttt{serenity.sh} located in the main folder (if \texttt{SERENITY\_USAGE\_FROM\_SOURCE} is set, which is the default) to set all necessary environment variables:
\begin{lstlisting}
cd ..
source serenity.sh
\end{lstlisting}
\section{Install Including Python Interface}
In order to activate the compilation of a Python interface to the code
a flag can be set as follows
\begin{lstlisting}
cmake -DSERENITY_PYTHON_BINDINGS=ON ..
\end{lstlisting}
Additionally this and other flags can be toggled using `ccmake`
\begin{lstlisting}
ccmake ..
\end{lstlisting}
The wrapper will be built for the Python version that CMake finds first.
In order to point CMake to a specific version of Python the following
option can be used:
\begin{lstlisting}
cmake -DPYTHON_EXECUTABLE=/usr/bin/python3 ..
\end{lstlisting}
The wrapper is shipped in form of a shared library (serenipy.so).
In order for Python to find this package the library folder has to be present
in the \texttt{PYTHONPATH} environment variable and the Serenity library has to be
present in a path searched by the system for shared libraries.
The latter is done when sourcing the \texttt{serenity.sh} script, the former requires
to un-comment one line in this file.
Afterwards the interface should importable as follows:
\begin{lstlisting}
python3
import serenipy as spy
\end{lstlisting}

\section{Installing Precompiled Binaries of the Python Interface}
  For CPython versions 3.7 - 3.13, binary wheels are available via \href{https://pypi.org/project/qcserenity/}{PyPI}. Just run
  \begin{lstlisting}
  pip install qcserenity
  \end{lstlisting}
  
  which installs the \texttt{qcserenity} Python package. Directly in this package are some Python utility functions. The pybind11 wrapper to the C++ code resides in a submodule of \texttt{qcserenity} called \texttt{serenipy}. Also see section \ref{sec:python}.

\section{Tests}
Serenity comes with a decent set of unittests; in order to run them, source
the \texttt{serenity.sh} script and run
\begin{lstlisting}
PATH_TO_BUILD_DIR/bin/serenity_tests
\end{lstlisting}
Please use the complete path, or else GTEST might run into problems.

The Python wrapper comes with its own set of tests, these can be run using
\begin{lstlisting}
python -m unittest discover PATH_TO_SERENITY/src/python/tests
\end{lstlisting}
\section{Documentation}
After configuring the project using CMake it is possible to create the documentation
using:
\begin{lstlisting}
make doc
\end{lstlisting}
Then you can open up the doc/html/index.html in a browser.
The Python wrapper is not featured inside the documentation.
Its documentation is available via Pythons help() function,
which displays the built-in doc-strings.

\clearpage
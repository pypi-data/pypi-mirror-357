\subsection{Task: Wavefunction Embedding Task}
\label{sec:WFinWFTask}
Allows the DLPNO-based coupled-cluster-in-coupled-cluster embedding according to Ref.~\cite{Sparta2017}.
\subsubsection{Example Input}
You can start the embedding calculation either from subsystems with previously calculated orbitals
(\ttt{fromFragments true}), or by partitioning a supersystem into subsystems based on the localization
of the supersystem orbitals on the subsystems (\emph{e.g.}, see the Top-Down embedding task in
Section~\ref{sec:topDownTask}).

Embedding with already calculated subsystem orbitals:
\begin{lstlisting}
+task WFEMB
  act supersystem
  env frag1
  env frag2
  env frag3
  fromFragments true
  fullDecomposition false
  +lc0
    method DLPNO-CCSD(T0)
    pnoSettings normal
    useFrozenCore true
  -lc0
  +lc1
    method DLPNO-CCSD
    pnoSettings loose
    useFrozenCore true
  -lc1
  +lc2
    method NONE
    useFrozenCore true
  -lc2
-task
\end{lstlisting}

Embedding by partitioing of the supersystem orbitals:
\begin{lstlisting}
+task WFEMB
  act supersystem
  env frag1
  env frag2
  fromFragments false
  fullDecomposition false
  +lc0
    method DLPNO-CCSD(T0)
    pnoSettings normal
    useFrozenCore true
  -lc0
  +lc1
    method DLPNO-CCSD
    pnoSettings loose
    useFrozenCore true
  -lc1
  +loc
    locType IBO
  -loc
  +split
    systemPartitioning bestmatch
  -split
-task
\end{lstlisting}

\subsubsection{Basic Keywords}
\begin{description}
    \item [\texttt{name}]\hfill \\
    Aliases for this task are \ttt{WFEMB} and \ttt{WAVEFUNCTIONEMBEDDING}.
	\item[\texttt{activeSystems}]\hfill \\
	The supersystems to be used.
	\item[\texttt{environmentSystems}]\hfill \\
	The subsystems to be used.
    \item [\texttt{sub-blocks}]\hfill \\
    The embedding (\ttt{emb}), local correlation (\ttt{lc}), PCM (\ttt{pcm}), localization task settings (\ttt{loc}) and
    system splitting task settings (\ttt{split}) are added via sub-blocks in the task settings.
    The local correlation settings are added as a list of settings. For each fragment (environment) system there is
    a local correlation settings object that determines the settings employed for the orbitals of this system. The
    task assumes that systems/settings with a lower index are tighter.
    By default \ttt{splitValenceAndCore = true} is set for the localization task settings.
    \item [\texttt{fullDecomposition}]\hfill \\
    If true, a full energy decomposition is calculated in terms of energy of the fragments and interaction energy.
    By default \ttt{false}.
    \item [\texttt{fromFragments}]\hfill \\
    If true, the supersystem is ignored and overwritten with the union of the subsystems. By default \ttt{false}.
\end{description}
\subsubsection{Advanced Keywords}
\begin{description}
  \item[\texttt{accurateInteraction}]\hfill\\
  If true, the settings with the lower subsystem index are used for cross system pairs.
  Otherwise: the settings with the higher subsystem index are used. By default \ttt{true}.
  \item [\texttt{maxCycles}]\hfill \\
  The maximum number of cycles allowed until convergence is expected by the coupled cluster iterations.
  By default 100 cycles are allowed.
  \item [\texttt{normThreshold}]\hfill \\
  The threshold at which the coupled cluster iterations are considered converged. By default \ttt{1.0e-5}.
  \item[\texttt{writePairEnergies}] \hfill \\
  If true, write the pair energies to a file with the name: systemName\_pairEnergies\_MultiLevelCC.dat. \ttt{False} by default.
\end{description}

\subsection{Task: Localization}\label{task:localization}
This task performs an orbital localization for the given active system. If the orbitals of
the active system are supposed to be aligned to the orbitals of a template system, the latter
has to be given as an environment system.

\subsubsection{Example Input}
\begin{lstlisting}
 +task LOC
   system water
   locType IBO
   splitValenceAndCore true
 -task
\end{lstlisting}
\subsubsection{Basic Keywords}
\begin{description}
 \item [\texttt{name}]\hfill \\
   Aliases for this task are \ttt{LocalizationTask}, \ttt{Localization} and \ttt{LOC}.
 \item [\texttt{activeSystems}]\hfill \\
   Accepts a single active system which will be used in the orbital localization
   (the orbitals are changed in place).
 \item [\texttt{environmentSystems}]\hfill \\
   Accepts a single environment systems that may be used as a template system for an orbital alignment.
   This is only relevant for \ttt{locType=ALIGN}.
 \item [\texttt{locType}]\hfill \\
   The orbital localization scheme to be used. The default is the intrinsic bond orbital scheme \ttt{IBO}.
   Other options are the  Pipek-Mezey (\ttt{PM}) scheme, the Foster--Boys (\ttt{FB}) scheme, the localization
   by Edminston and Ruedenberg (\ttt{ER}), a localization that does not retain orbital orthogonality
   (\ttt{NO}) and an orbital alignment procedure that aligns the orbitals to a given template system \ttt{ALIGN}\cite{Bensberg2020}.
 \item [\texttt{splitValenceAndCore}]\hfill \\
   If \ttt{true}, the valence and core orbitals are localized independently (no mixing). This is useful
   for local correlation methods. Furthermore, core like orbitals will resemble e.g. their $1s$ character
   more closely. The default is \ttt{false}. If set to true and \texttt{useEnergyCutOff = false}, core
   orbitals will be determined using tabulated  numbers of core orbitals for each element, selecting always
   the lowest orbitals according to their energy eigenvalue. If desired an energy cut-off
   (see \texttt{useEnergyCutOff}) can be used instead or a number of core orbitals may be specified manually.
   Note that the information about core orbitals is kept between tasks.
 \item [\texttt{localizeVirtuals}]\hfill \\
   If true, the virtual orbitals are localized as well. By default \ttt{false}. This is only supported for the IBO and ALIGN schemes.
\end{description}
\subsubsection{Advanced Keywords}
\begin{description}
 \item [\texttt{maxSweeps}]\hfill \\
   Most orbital localization schemes are iterative, this is the maximum number of cycles allowed.
   The default is $100$.
 \item [\texttt{useEnergyCutOff}]\hfill \\
   If true, core orbitals are determined using an energy cut-off. By default \ttt{true}.
   Needs \texttt{splitValenceAndCore = true}.
 \item [\texttt{energyCutOff}]\hfill \\
    The energy cut-off used for the core orbital selection. The default is $-5.0$ au.
 \item [\texttt{nCoreOrbitals}]\hfill \\
    Use a predefined number of core orbitals. Needs \texttt{useEnergyCutOff = false} and \texttt{splitValenceAndCore = true}.
    By default this is not used, \emph{i.e.}, the number of orbitals is set to \ttt{infinity}.
 \item [\texttt{useKineticAlign}]\hfill \\
   If \ttt{true}, the orbital kinetic energy is used as an additional criterion in the orbital alignment.
   The default is \ttt{false}.
 \item [\texttt{alignExponent}]\hfill \\
   The exponent for the penalty function used in the orbital alignment. This has to be an even integer
   larger or equal to $2$. The default is $4$. Exponents of $8$ or larger are not recommended since they
   may lead to numerical instabilities.
 \item [\texttt{splitVirtuals}]\hfill \\
   If true, the valence virtuals and diffuse virtuals are localized separately. By default \ttt{true}.
 \item [\texttt{virtualEnergyCutOff}]\hfill \\
   Orbital eigenvalue threshold to select diffuse virtual orbitals. By default $1.0$.
 \item [\texttt{nRydbergOrbitals}]\hfill \\
   Use a predefined number of diffuse virtual orbitals. Not used by default.
 \item [\texttt{replaceVirtuals}]\hfill \\
   Reconstruct the virtual orbitals before localization by projecting all occupied orbitals
   and cleanly separating valence virtuals from diffuse virtuals. This should be used if the
   \ttt{IBO} or \ttt{ALIGN} approaches are chosen and the same orbital set was not already reconstructed
   in this manner before. By default \ttt{false}.
\end{description}

\chapter{Short History of Serenity Until the First Release}

The work on \serenity started originally as a hobby project
by Thomas Dresselhaus (TD). The first
documented commit by TD, at the time a PhD student in the group of
Johannes Neugebauer (JN), is from March 24, 2013. During 2013,
TD extended this project into a restricted HF-SCF code
(working with up to $d$-functions).\\
\\
Accompanied by continuous discussions with and support from the
research group head (JN), the project gained
momentum in 2014. During a first hackathon with TD, Jan Unsleber (JU),
and Dennis Barton (DB) in January 2014, the code was extended to a
restricted DFT program. In fall 2014, when the working title of the
project was already changed to Serenity, TD (and JU) initiated a
\serenity Seminar in the Neugebauer group, in which further work
on the program was discussed and planned. Also about this time,
first \serenity-related projects were integrated into PhD projects
in the Neugebauer group.\\
\\
With Kevin Klahr (KK, 02/2015), David Schnieders (DS, 11/2015), and Michael
Boeckers (MB, 02/2016), three more PhD students joined the SereniTEAM and
the program developed into a multipurpose embedding code.
The program, which received its official release name in March 2015, and
which was moved to a Git repository on a self-hosted GitLab server in
November 2015, reached a status that allowed to use it as a coding
platform for Bachelor and Master thesis work in early 2016. By 11/2017,
10 BSc theses and 6 MSc research projects/theses had employed it as
a code basis.\\
\\
In the course of 2016, JU effectively evolved into TD's successor as the
lead developer of the code and shaped its current structure. He also
coordinated the work towards the first release version, which was mainly
planned by himself, TD, and the group head (JN) from fall 2016 onwards.
The first release candidate of Serenity could be distributed to specific
testers in 2017. The release of version 1.0.0 took place in December 2017.